\documentclass{pracamgr}
\usepackage{polski}

\usepackage[utf8]{inputenc}
\usepackage[OT4]{fontenc}

\usepackage[pdftex]{graphicx}
\usepackage{wrapfig}
\usepackage{subfig}
\usepackage{amsmath}
\usepackage{enumerate}
% Dane magistranta:

\author{Małgorzata Habich}

\nralbumu{280454}

\title{Systemowa analiza reakcji drożdży na szok cieplny}

\tytulang{---> tytuł ang <---} %TODO

%kierunek: Matematyka, Informatyka, ...
\kierunek{Bioinformatyka i biologia systemów}

% informatyka - nie okreslamy zakresu (opcja zakomentowana)
% matematyka - zakres moze pozostac nieokreslony,
% a jesli ma byc okreslony dla pracy mgr,
% to przyjmuje jedna z wartosci:
% {metod matematycznych w finansach}
% {metod matematycznych w ubezpieczeniach}
% {matematyki stosowanej}
% {nauczania matematyki}
% Dla pracy licencjackiej mamy natomiast
% mozliwosc wpisania takiej wartosci zakresu:
% {Jednoczesnych Studiow Ekonomiczno--Matematycznych}

% \zakres{Tu wpisac, jesli trzeba, jedna z opcji podanych wyzej}

% Praca wykonana pod kierunkiem:
% (podaæ tytu³/stopieñ imiê i nazwisko opiekuna
% Instytut
% ew. Wydzia³ ew. Uczelnia (je¿eli nie MIM UW))
\opiekun{dr Paweł Szczęsny\\
  --> INSTYTUT <--- \\%TODO
  }

% miesi±c i~rok:
\date{Czerwiec 2013}

%Podaæ dziedzinê wg klasyfikacji Socrates-Erasmus:
\dziedzina{ 
%11.0 Matematyka, Informatyka:\\ 
%11.1 Matematyka\\ 
%11.2 Statystyka\\ 
%11.3 Informatyka\\ 
%11.4 Sztuczna inteligencja\\ 
%11.5 Nauki aktuarialne\\
11.9 Inne nauki matematyczne i informatyczne
}

%Klasyfikacja tematyczna wedlug AMS (matematyka) lub ACM (informatyka)
\klasyfikacja{--> KLASYFIKACJA <--} %TODO

% S³owa kluczowe:
\keywords{--> SŁOWA KLUCZOWE <-- }%TODO

% Tu jest dobre miejsce na Twoje w³asne makra i~¶rodowiska:
\newtheorem{defi}{Definicja}[section]

% koniec definicji

\begin{document}
\maketitle

%tu idzie streszczenie na strone poczatkowa
\begin{abstract}
--> ABSTRAKT <--
%TODO
\end{abstract}

\tableofcontents
%\listoffigures
%\listoftables

\chapter*{Wprowadzenie}
\addcontentsline{toc}{chapter}{Wprowadzenie}


\section{Temperatura jako czynnik stresowy}

Wzrost temperatury stanowi uniwersalny czynnik stresowy w świecie komórkowym. 
 %TODO


1. Uniwersalny sygnał
2. Dlaczego temperatura tak działa
3. Co się zmienia na poszczególnych poziomach organizacji i jakie to ma konsekwencje
-> denaturacja białek
-> przyspieszenie reakcji enzymatycznych 
-> akumulacja związków
-> problemy z błoną komórkową


% bialka wrazliwe
% duzo rzeczy sie psuje
% jakie szkody


\section{Reakcja na temperaturę u drożdży}
% do poziomu komorkowego ok 2 strony

\section{Sensory reakcji na stres temperaturowy}
%chaperony (?)
%sensory powierzchniowe
% termometry RNA
% termometry bialkowe




\chapter{Cel pracy}
% nie ma oczywistego sensora w ujeciu komorkowym przesuniec metabolicznych wywolywanych zmianami km/T
% integrancja sensora z reszta systemu

\chapter{Metody}

\chapter{Wyniki i dyskusja}

\chapter{Podsumowanie}



\begin{thebibliography}{99}
% minimum 50 pozycji

\end{thebibliography}

\end{document}


%%% Local Variables:
%%% mode: latex
%%% TeX-master: t
%%% coding: latin-2
%%% End:
