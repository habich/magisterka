\documentclass{pracamgr}
\usepackage{polski}

\usepackage[utf8]{inputenc}
\usepackage[OT4]{fontenc}

\usepackage[pdftex]{graphicx}
\usepackage{wrapfig}
\usepackage{subfig}
\usepackage{amsmath}
\usepackage{enumerate}
\usepackage{textcomp}
% Dane magistranta:

\author{Małgorzata Habich}

\nralbumu{280454}

%\title{Systemowa analiza reakcji drożdży na szok cieplny}
\title{Globalny metabolizm jako sensor stresu - analiza odpowiedzi na temperaturę u drożdży}

\tytulang{---> tytuł ang <---} %TODO

%kierunek: Matematyka, Informatyka, ...
\kierunek{Bioinformatyka i biologia systemów}

% informatyka - nie okreslamy zakresu (opcja zakomentowana)
% matematyka - zakres moze pozostac nieokreslony,
% a jesli ma byc okreslony dla pracy mgr,
% to przyjmuje jedna z wartosci:
% {metod matematycznych w finansach}
% {metod matematycznych w ubezpieczeniach}
% {matematyki stosowanej}
% {nauczania matematyki}
% Dla pracy licencjackiej mamy natomiast
% mozliwosc wpisania takiej wartosci zakresu:
% {Jednoczesnych Studiow Ekonomiczno--Matematycznych}

% \zakres{Tu wpisac, jesli trzeba, jedna z opcji podanych wyzej}

% Praca wykonana pod kierunkiem:
% (podaæ tytu³/stopieñ imiê i nazwisko opiekuna
% Instytut
% ew. Wydzia³ ew. Uczelnia (je¿eli nie MIM UW))
\opiekun{dra Pawła Szczęsnego\\
  --> INSTYTUT <--- \\%TODO
  }

% miesi±c i~rok:
\date{Czerwiec 2013}

%Podaæ dziedzinê wg klasyfikacji Socrates-Erasmus:
\dziedzina{ 
%11.0 Matematyka, Informatyka:\\ 
%11.1 Matematyka\\ 
%11.2 Statystyka\\ 
%11.3 Informatyka\\ 
%11.4 Sztuczna inteligencja\\ 
%11.5 Nauki aktuarialne\\
11.9 Inne nauki matematyczne i informatyczne
}

%Klasyfikacja tematyczna wedlug AMS (matematyka) lub ACM (informatyka)
\klasyfikacja{--> KLASYFIKACJA <--} %TODO

% S³owa kluczowe:
\keywords{--> SŁOWA KLUCZOWE <-- }%TODO

% Tu jest dobre miejsce na Twoje w³asne makra i~¶rodowiska:
\newtheorem{defi}{Definicja}[section]

% koniec definicji

\begin{document}
\maketitle

%tu idzie streszczenie na strone poczatkowa
\begin{abstract}
--> ABSTRAKT <--
%TODO
\end{abstract}

\tableofcontents
%\listoffigures
%\listoftables

\chapter{Wprowadzenie}

Znaczenie drozdzy \cite{100years}
Tolerancja na stres kluczowa dla produkcji drozdzy \cite{Stresstolerance}


\section{Temperatura jako czynnik stresowy}

Temperatura stanowi uniwersalny czynnik stresowy w świecie komórkowym. Pomijając nieliczne organizmy należące do grupy hipertermofili, wzrost temperatury stawnowi poważne zagrożenie
dla prawidłowego funkcjonowania i życia komórki. Dzieje się tak dlatego, że temperatura oddziałowuje bezpośrednio na wszystkie elementy systemu zaburzając homeostazę. 
Z punktu widzenia termodynamiki wzrost ciepła wiąże się z wzmorzonym ruchem atomów. W wyniku tego rozrywają się wiązania krótkiego zasięgu (tj. wiązania wodorowe i oddziaływania Van der Waalsa) oraz 
zwiększa się ruchliwość cząsteczek. Złożenie tych dwóch zjawisk ma dla komórki bardzo szerokie skutki.

Jednym z największych problemów z jakimi musi sobie poradzić komórka podczas wzrostu temperatury jest denaturacja białek. Funkcja tych związków wiąże się bezpośrednio z ich kształtem za który odpowiadają 
wiązania wodorowe podtrzymujące strukturę drugorzędową. Gdy wzrasta temperatura wiązania rozrywają się i najczęściej w sposób nieodwracalny zaburzana jest struktura białka. Denaturacja niesie za sobą
dwa problemy z punktu widzenia komórki. Z jednej strony tracą one niezbędne do funkcjonowania składniki które trzeba uzupełnić, a z drugiej strony zdenaturowane białka mają tendencję
do agregacji przez co zajmują miejsce i utrudniają swoją degradację. Należy jednak pamiętać, że nie wszystkie białka są równie czułe na temperaturę. Dzięki temu możliwe jest wczesne wykrycie wzrostu temperatury (białka
wyjątkowo czułe) oraz przeżycie podczas warunków stresowych (najcześciej białka które pojawiają się podczas wzrostu temperaury mają budowę bardziej odporną na działanie tego czynnika)\cite{}.

Rozrywanie się wiązań wodorowych niesie ze sobą również inne konsekwencje. Wiązania wodorowe podtrzymują strukturę RNA i DNA. Poprzez działanie temperatury może dochodzić do superskręcania się DNA co uniemożliwia 
eksperesję genów oraz rozklejania się specyficznych struktur RNA. Podobnie jak w przypadku białek część organizmów umie to obrócić na swoją korzyść. W przypadku E.coli system wyczuwania zmian temperatury opiera się
w dużej mierze na różnicach w topnieniu RNA i DNA\cite{TsInEubact} \cite{Digel08}.

Innym problemem związanym z rozluźnianiem się wiązań jest zwiększona płynność błony komórkowej. Niestety nie jest to zbyt dobrze zbadany problem\cite{Membranefluidity}, ale obecne wyniki pozwalają zaobserwować zwiększoną
płynność błony komórkowej, rozluźnienie jej struktury, a tym samym większą przepuszczalność. Z uwagi na bardzo groźne konsekwencje dla komórki reakcja systemu jest bardzo szybka i prowadzi do zagęszczenia błony.

Nie tylko rozrywanie się wiazań wodorowych stanowi problem dla komórki. Podwyższenie się temperatury prowadzi do zwiększenia się szybkości reakcji enzymatycznych. Zgodnie z regułą $Q_{10}$ przeciętny wzrost
szybkości reakcji wzrasta dwukrotnie przy zmianie temperatury o 10 \textcelsius, aż do gwałtownego spadku związanego z denaturacją białka\cite{}. Niekontrolowane zwiększenie się szybkości reakcji enzymatycznych
może prowadzić do nagromadzenia się produktów, nieoptymalnego wydatku energii i zaburzeniu homeostazy. W badaniach przeprowadzonych na bakteriach zaobserwowano wzrost szybkości glikolizy, spadek przepływów przez 
cykl Krebsa oraz nagromadzenie się XXXXXX \cite{Wittmann07}

Patrząc na komórkę jako na całość można zaobserwować również wpływ temperatury na jej cykl życiowy. Pod wpływem wielu zmian które następują w komórce w wielu organizmach jednokomórkowych np. drożdżach 
następuje zatrzymanie się w fazie G$_1$.

\section{Reakcja na temperaturę u drożdży}

Optymalna temperatura wzrostu drożdży różni się w zależności od szczepu, ale przyjmuje się, że są to temperatury w 
zakresie między 25 \textcelsius, a 30 \textcelsius. Powyżej 36 \textcelsius\ uruchamiany jest system ochrony komórki
zwany HSP (heat shock response). Maksymalną temperaturą  w której mogą rozwijać się jest 42 \textcelsius\ powyżej której 
następuje inaktywacja polimerazy II RNA \cite{Morano12}.

Tradycyjne doświadczenie polegające na badaniu reakcji drożdży na temperaturę polega na inkubacji komórek w 30 \textcelsius, a
następnie przeniesienia ich do 37 \textcelsius. Wykonywane były oczywiście również inne dobory parametrów, ponieważ 
reakcja różni się w zależności od czasu ekspozycji na temperaturę, jak często następuje podgrzanie, jak szybko zmieniana
jest temperatura itd.

Już w latach 80' zaobserwowano, że ekspozycja drożdży na działanie temperatury wywołuje pojawienie się białek których 
wcześniej nie było w proteomie komórki \cite{Miller1979}, a wcześniejsza inkubacja w wyższej temperaturze pozwala 
na uniknięcie śmierci w ekstremalnych warunkach 50 \textcelsius \cite{Mcalister1980}.

Reakcję drożdży na temperaturę można rozpatrywać na kilku poziomach złożności. Patrząc z punktu widzenia makro na komórkę
można zaobserwować zmiany w cyklu życiowym. Organizm poddany szoku termicznemu zatrzymuje się w fazie G$_1$. Dzieje się tak 
z uwagi na redukcję tranksryptów, oraz postraslacyjne modyfikacje cyklin fazy G$_1$/S CLN1 oraz CLN2\cite{CyclinArrest}. Jest to prawdopodobnie
wywołane nagromadzeniem się zdenaturowanych białek, co pośrednio aktywuje czynnik transkrypcyjny HSF1\cite{MisfoldedProteins}. Można się domyślać, że 
podział komórki w niesprzyjających okolicznościach mógłby mieć dla niej katastrofalne skutki.	

Komórki drożdży są ograniczone od świata zewnętrznego poprzez błonę komórkową i ścianę złożoną z glukozo- i maltozo- polisacharydów
 i N-acetyloglukozoamin. Podczas swojego życia utrzymują one wysokie ciśnienie wewnętrzene sprawiając, że nawet małe defekty w ścianie 
 są potencjalnie letalne. W przestrzeni periplazmatycznej znajduje się wiele pierwotnych recetorów na stres w tym termalny. 
 Pod wpływem temperatury uruchamiają się szlaki sygnalizacji komórowej co prowadzi do zwiększonej produkcji $\beta$1,6-glukanu
 oraz białek CWP (cell wall proteins)\cite{CellWall}. Ponad to zmniejsza się ilość nienasyconych kwasów tłuszczowych w błonach \cite{CellLipids}
 dzięki czemu równoważona jest zwiększona płynność błony.
 
 Istotna zmiana następuje w metabolizmie komórki
 %TODO ZMIANA TEMATU => ZMIANA ZDANIA
 . Mimo, że nie jest znany dokładny mechanizm reakcji na komórki na zwiększone 
 tempo reakcji enzymatycznych, to moźna zaobserwować inne ciekawe zmiany. Największe widać w reakcjach odpowiedzialnych za
 XXX cukrów. Pod wpływem podwyższonej temperatury następuje akumulacja glicerolu \cite{GlycerolAccumulate} XXXXXX (9.pdf)oraz
 zwiększa się produkcja trehalozy. Dwucukier występuje również w warunkach normalnych w komórce, lecz w blisko sto razy mniejszym stężeniu. 
 Jego zadaniem jest chronić białka przed denaturacją i po obniżeniu temperatury jego poziom wraca do pierwotnego stanu. Inną obserwowaną
 zmianą jest podwyższenie się ilości cAMP. Jest to o tyle interesujące, że poziom tego związku wpływa na aktywację PKA które negatywnie
 reguluje działanie czynnika transkrypcyjnego Msn2/4 odpowiedzialnego za większość reakcji temperaturowych. Prawdopodobnie 
 jest to wewnętrzny hamulec systemu nie pozwalajacy na przesadzoną reakcję na stres\cite{Blomberg00,StressResponse99}.

 Typową obserwacją w komórkach wystawionych na stres cieplny jest nagromadzenie się źle sfałdowanych białek. 
 W przypadku drożdży taka reakcja pojawia się dopiero przy temperaturach ekstremalnych, powyżej 50 \textcelsius. Można
 wtedy obserwować zarówno toksyczną agregację zdenaturowanych białek jak i groźne dla funkcjonowania komórki ubytki w 
 ilości białek niezbędnych do życia. Podczas umiarkowanego szoku termicznego (ok. 36-37 \textcelsius) nie obserwuje się
 agregatów. Prawdopodobnie wynika to z bardzo sprawnej maszynerii odpowiedzi na stres cieplny, ubikwitynacji zdenaturowanych
 białek i ich degradacji\cite{Bible}.
 

 Największą przemianą jaka następuje w komórce podczas reakcji na temperaturę jest zmiana profilu białkowego. Za większość tej 
 reakcji odpowiadają czynnki transkrypcyjne Hsf1 i Msn2/4. 
%TODO
 
 7.pdf
 trehaloza YingYangTrehalose
 
 Glikoliza Control of Glycolytic Oscillations by Temperautr Mair et all
 
Problemy z DNA i reakcja na temperaturę są oddzielnie kontrolowane \cite{Duallyregulated85}

Regulacja transkrypcji \cite{Yamamoto08}

The Response to Heat Shock and Oxidative Stress
in Saccharomyces cerevisiae \cite{Morano12}

Reakcja cala \cite{Bible}
 

Zmiany profilu ekspresji \cite{Gash00}

1. Problemy z RNA/DNA

% do poziomu komorkowego ok 2 strony

\section{Sensory reakcji na stres temperaturowy}

W przeciwieństwie do innych czynników stresowych temperatura oddziałuje praktycznie w tym samym czasie na całą komórkę. 
Dzięki temu sensory tego bodźca możemy znaleźć nie tylko w formie białek zwieszonych w błonie komórkowej, ale również 
pod postacią mRNA czułego na temperaturę, białek o niskiej temperaturze denaturacji, czy wreszcie odpowiedzi metabolizmu.

Wydaje się, że pierwszymi sensorami temperatury są białka błonowe które uruchamiają dalsze kaskady sygnalizacyjne. Do 
najlepiej poznanych należy ścieżka aktywacji HOG której receptorem jest białko błonowe Sho1, a końcowym wynikiem 
aktywacja czynników transkrypcyjnych z rodziny Msn\cite{SensingLesson}. Innym często wymienianym w literaturze szlakiem sygnałowym odpowiedzialnym 
za reakcję na temperaturę jest szlak CWI (cell wall integrity) który odpowiada za czynniki traskrypcyjne wspierające 
naprawę ściany komórkowej. \cite{CWI} %TODO 5 wpisac
 Co ciekawe błona komórkowa nie jest jedynie biernym organellum które chroni
 komórkę i służy jako rusztowanie dla sensorów, ale również sama jej płynność stanowi informację dla komórki o szoku cieplnym.
 Dynamika błony komórkowej ma wpływ na reakcję HSR\cite{Carratu96}

Stosunkowo szybką reakcją jest sposób sygnalizacji wynikający z nagromadzenia się zdenaturowanych białek. Dzięki tej ścieżce
sygnałowej aktywowany jest czynnik transkrypcyjny Hsf1 - najbardziej charakterystyczny wyznacznik szoku cieplnego. Dokładny mechanizm
sygnalizacji jest jednak niejasny i w sposób bardziej szczegółowy przedstawie go w dalszej części pracy.

Kolejnym typem sygnalizacji są termometry RNA. Jest to co prawda bardziej wtórny sposób sygnalizacji, ponieważ dotyczy się
transkryptów białek które mają dopiero powstać, lecz wpływa ona również na kształtowanie się odpowiedzi komórki na stres cieplny.
Jak wynika z badań białka o kluczowej roli dla reakcji na szok temperaturowy mają wyższą temperaturę topnienia mRNA 
Termometry RNA \cite{RNAterm}
Stabilność RNA \cite{Roca11}

Niezbyt dokladnie

 
%chaperony (?)
%sensory powierzchniowe
% termometry RNA
% termometry bialkowe




\chapter{Cel pracy}

Probuje zobaczyc jaka jest dokladna droga sygnalizacji temperatury. po pierwsze sa bialka blonowe ktore sa dobrze opisane ale ja to zrobilam lepiej
po drugie sa reakcje bialek po trzecie jest rna po czwarte jest odpowiedz z metabolizmu. generalnie odpowiedz metaboolizmu jest kompenosowana ale pytanie w jaki sposob.
sa uklady autokatalityczne ktore sa wrazliwe na wahania temperatury. a co ze statystyczna zmiana szybkosci reakcji wynikajacych ze zmiany km od temperatury?
nikt tego wczesniej nie ruszal wiec aby opracowac pelny model sygnalizacji (bo z pewnoscia metabolizm od razu reaguje) trzeba zajac sie tym.
% nie ma oczywistego sensora w ujeciu komorkowym przesuniec metabolicznych wywolywanych zmianami km/T
% integrancja sensora z reszta systemu

\chapter{Materiały i Metody}

1. budowa modelu
-> program X 
2. Badanie odpowiedzi glikolizy
-> program Y
3. Dane na temat 

\chapter{Wyniki i dyskusja}

\chapter{Podsumowanie}

Kompensacja odpowiedzi metabolizmu nie tylko jako zlozenie wyhylen autokatalitycznych ale takze zwyklych wychylen glikolizy wynikajacych ze zmian km


\bibliographystyle{amsplain}
\bibliography{bibliography}

\end{document}


