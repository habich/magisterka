\documentclass{pracamgr}
\usepackage{polski}

\usepackage[utf8]{inputenc}
\usepackage[OT4]{fontenc}

\usepackage[pdftex]{graphicx}
\usepackage{wrapfig}
\usepackage{subfig}
\usepackage{amsmath}
\usepackage{enumerate}
\usepackage{textcomp}
% Dane magistranta:

\author{Małgorzata Habich}

\nralbumu{280454}

\title{Systemowa analiza reakcji drożdży na szok cieplny}

\tytulang{---> tytuł ang <---} %TODO

%kierunek: Matematyka, Informatyka, ...
\kierunek{Bioinformatyka i biologia systemów}

% informatyka - nie okreslamy zakresu (opcja zakomentowana)
% matematyka - zakres moze pozostac nieokreslony,
% a jesli ma byc okreslony dla pracy mgr,
% to przyjmuje jedna z wartosci:
% {metod matematycznych w finansach}
% {metod matematycznych w ubezpieczeniach}
% {matematyki stosowanej}
% {nauczania matematyki}
% Dla pracy licencjackiej mamy natomiast
% mozliwosc wpisania takiej wartosci zakresu:
% {Jednoczesnych Studiow Ekonomiczno--Matematycznych}

% \zakres{Tu wpisac, jesli trzeba, jedna z opcji podanych wyzej}

% Praca wykonana pod kierunkiem:
% (podaæ tytu³/stopieñ imiê i nazwisko opiekuna
% Instytut
% ew. Wydzia³ ew. Uczelnia (je¿eli nie MIM UW))
\opiekun{dra Pawła Szczęsnego\\
  --> INSTYTUT <--- \\%TODO
  }

% miesi±c i~rok:
\date{Czerwiec 2013}

%Podaæ dziedzinê wg klasyfikacji Socrates-Erasmus:
\dziedzina{ 
%11.0 Matematyka, Informatyka:\\ 
%11.1 Matematyka\\ 
%11.2 Statystyka\\ 
%11.3 Informatyka\\ 
%11.4 Sztuczna inteligencja\\ 
%11.5 Nauki aktuarialne\\
11.9 Inne nauki matematyczne i informatyczne
}

%Klasyfikacja tematyczna wedlug AMS (matematyka) lub ACM (informatyka)
\klasyfikacja{--> KLASYFIKACJA <--} %TODO

% S³owa kluczowe:
\keywords{--> SŁOWA KLUCZOWE <-- }%TODO

% Tu jest dobre miejsce na Twoje w³asne makra i~¶rodowiska:
\newtheorem{defi}{Definicja}[section]

% koniec definicji

\begin{document}
\maketitle

%tu idzie streszczenie na strone poczatkowa
\begin{abstract}
--> ABSTRAKT <--
%TODO
\end{abstract}

\tableofcontents
%\listoffigures
%\listoftables

\chapter{Wprowadzenie}

Znaczenie drozdzy \cite{100years}
Tolerancja na stres kluczowa dla produkcji drozdzy \cite{Stresstolerance}


\section{Temperatura jako czynnik stresowy}

Temperatura stanowi uniwersalny czynnik stresowy w świecie komórkowym. Pomijając nieliczne organizmy należące do grupy hipertermofili, wzrost temperatury stawnowi poważne zagrożenie
dla prawidłowego funkcjonowania i życia komórki. Dzieje się tak dlatego, że temperatura oddziałowuje bezpośrednio na wszystkie elementy systemu zaburzając homeostazę. 
Z punktu widzenia termodynamiki wzrost ciepła wiąże się z wzmorzonym ruchem atomów. W wyniku tego rozrywają się wiązania krótkiego zasięgu (tj. wiązania wodorowe i oddziaływania Van der Waalsa) oraz 
zwiększa się ruchliwość cząsteczek. Złożenie tych dwóch zjawisk ma dla komórki bardzo szerokie skutki.

Jednym z największych problemów z jakimi musi sobie poradzić komórka podczas wzrostu temperatury jest denaturacja białek. Funkcja tych związków wiąże się bezpośrednio z ich kształtem za który odpowiadają 
wiązania wodorowe podtrzymujące strukturę drugorzędową. Gdy wzrasta temperatura wiązania rozrywają się i najczęściej w sposób nieodwracalny zaburzana jest struktura białka. Denaturacja niesie za sobą
dwa problemy z punktu widzenia komórki. Z jednej strony tracą one niezbędne do funkcjonowania składniki które trzeba uzupełnić, a z drugiej strony zdenaturowane białka mają tendencję
do agregacji przez co zajmują miejsce i utrudniają swoją degradację. Należy jednak pamiętać, że nie wszystkie białka są równie czułe na temperaturę. Dzięki temu możliwe jest wczesne wykrycie wzrostu temperatury (białka
wyjątkowo czułe) oraz przeżycie podczas warunków stresowych (najcześciej białka które pojawiają się podczas wzrostu temperaury mają budowę bardziej odporną na działanie tego czynnika)\cite{}.

Rozrywanie się wiązań wodorowych niesie ze sobą również inne konsekwencje. Wiązania wodorowe podtrzymują strukturę RNA i DNA. Poprzez działanie temperatury może dochodzić do superskręcania się DNA co uniemożliwia 
eksperesję genów oraz rozklejania się specyficznych struktur RNA. Podobnie jak w przypadku białek część organizmów umie to obrócić na swoją korzyść. W przypadku E.coli system wyczuwania zmian temperatury opiera się
w dużej mierze na różnicach w topnieniu RNA i DNA\cite{TsInEubact} \cite{Digel08}.

Innym problemem związanym z rozluźnianiem się wiązań jest zwiększona płynność błony komórkowej. Niestety nie jest to zbyt dobrze zbadany problem\cite{Membranefluidity}, ale obecne wyniki pozwalają zaobserwować zwiększoną
płynność błony komórkowej, rozluźnienie jej struktury, a tym samym większą przepuszczalność. Z uwagi na bardzo groźne konsekwencje dla komórki reakcja systemu jest bardzo szybka i prowadzi do zagęszczenia błony.

Nie tylko rozrywanie się wiazań wodorowych stanowi problem dla komórki. Podwyższenie się temperatury prowadzi do zwiększenia się szybkości reakcji enzymatycznych. Zgodnie z regułą $Q_{10}$ przeciętny wzrost
szybkości reakcji wzrasta dwukrotnie przy zmianie temperatury o 10 \textcelsius, aż do gwałtownego spadku związanego z denaturacją białka\cite{}. Niekontrolowane zwiększenie się szybkości reakcji enzymatycznych
może prowadzić do nagromadzenia się produktów, nieoptymalnego wydatku energii i zaburzeniu homeostazy. W badaniach przeprowadzonych na bakteriach zaobserwowano wzrost szybkości glikolizy, spadek przepływów przez 
cykl Krebsa oraz nagromadzenie się XXXXXX \cite{Wittmann07}

Patrząc na komórkę jako na całość można zaobserwować również wpływ temperatury na jej cykl życiowy. Pod wpływem wielu zmian które następują w komórce w wielu organizmach jednokomórkowych np. drożdżach 
następuje zatrzymanie się w fazie G$_1$.

\section{Reakcja na temperaturę u drożdży}

Z uwagi na przemyslowe znaczenie Saccharomyces cerevisiae\cite{100years,Legras2007} ich zakresy tolerancji temperaturowej zostaly dobrze zbadane.
Dokladne wartości różnią się w zależności od szczepu, ale przyjmuje się, że najlepsza temperatura wzrostu wynosi DANE, a szok temperaturowy zaczy na się od DANE \textcelsius.

A rapid shift in cultivation temperature of View the MathML source from 23° to 36° results in protection from death due to extreme heat treatment (52°).
The level of this acquired thermal resistance shows an excellent correlation with 
the cellular level of the heat shock proteins which are transiently induced by such a temperature shift.\cite{Mcalister1980}

Wzrost syntezy związków których wcześniej nie było \cite{Miller1979}

Problemy z DNA i reakcja na temperaturę są oddzielnie kontrolowane \cite{Duallyregulated85}

Regulacja transkrypcji \cite{Yamamoto08}

The Response to Heat Shock and Oxidative Stress
in Saccharomyces cerevisiae \cite{Morano12}

Glicerol/trehaloza/glikoliza \cite{Blomberg00}

Dynamika budowy błony komórkowej ma wpływ na reakcję HSP\cite{Carratu96}

Reakcja cala \cite{Bible}

Sposoby wyczuwania \cite{SensingLesson}

Termometry RNA \cite{RNAterm}
Stabilność RNA \cite{Roca11}

Zmiany profilu ekspresji \cite{Gash00}

1. Problemy z RNA/DNA

2. Denaturacja

3. Agregacja białek 

4. Zatrzymanie cyklu komórkowego

5. Zwiększenie szybkości reakcji enzymatycznych





% do poziomu komorkowego ok 2 strony

\section{Sensory reakcji na stres temperaturowy}
%chaperony (?)
%sensory powierzchniowe
% termometry RNA
% termometry bialkowe




\chapter{Cel pracy}


% nie ma oczywistego sensora w ujeciu komorkowym przesuniec metabolicznych wywolywanych zmianami km/T
% integrancja sensora z reszta systemu

\chapter{Metody}

\chapter{Wyniki i dyskusja}

\chapter{Podsumowanie}



\bibliographystyle{amsplain}
\bibliography{bibliography}

\end{document}


%%% Local Variables:
%%% mode: latex
%%% TeX-master: t
%%% coding: latin-2
%%% End:
